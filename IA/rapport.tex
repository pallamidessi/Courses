\documentclass{article}
\usepackage[francais]{babel}
\usepackage[utf8]{inputenc}
\begin{document}

\title{Rapport de TP d'IA}
\author{Pallamidessi Joseph 
}
\maketitle
\newpage

\section{Intro} % (fold)
\label{sec:Intro}
\paragraph{} % (fold)
\label{par:}

Le sujet a été realisé dans son integralité. Les valeurs trouvées par le programme ont été
verifié grâce à des outils externes tel que : identify de la suite imagemagick, pour les
valeurs de l'histogramme et des scripts de calcul d'entropie. Le TP a été écrit en C++ et
utilise bien le paradigme objets.
% paragraph  (end)

\paragraph{} % (fold)
\label{par:}
Il n'y a pas de fuite mémoire (vérification avec valgrind), ou d'erreurs quelconques, ni à la
compilation, ni à l'execution.\\
Le code est entièrement commenté. Je m'excuse cependant du faite qu'une partie des
commentaires soit en anglais et l'autre en francais.
% paragraph  (end)
% section Intro (end)

\section{Mise en place} % (fold)
\label{sec:Mise en place}
\paragraph{} % (fold)
\label{par:}
Il suffit de faire un simple make pour compiler les TPs. Une commande make clean est
également disponible.\\
Le programme se lance de la façon suivante: dans le répertoire bin, ./main [nom du bmp en
niveau de gris].\\
% paragraph  (end)
\paragraph{} % (fold)
\label{par:}
Le programme va alors calculer l'entropie de l'image, puis rechercher de manière exhaustive
le meilleur couple, puis enfin lancer l'algorithme génétique de recherche du meilleurs
couple.
% paragraph  (end)

\section{Remarques diverses} % (fold)
\label{sec:Remarque}
\paragraph{} % (fold)
\label{par:}

  La recherche exhaustive fonctionne bien et est relativement rapide, la valeurs trouvée
  est cohérente et constante selon l'image (algo déterministe).\\
  La recheche par l'algorithme génétique est elle aussi cohérente et efficace, le couple
  trouvé au bout de 100 génération est des fois le même que celui de l'algorithme
  exhaustif, ou s'il est différent, a une entropie très proche.

  \paragraph{} % (fold)
  \label{par:}
  Il n'y a malheuresement pas de détection de la convergence, il est donc fortement
  possible que la recherche génétique finisse (converge) bien avant les 100 générations
  imposées.
  % paragraph  (end)

% paragraph  (end)
% section Fonstionnement et remarque  (end)
% section Mise en place (end)

\end{document}
\end{article}
