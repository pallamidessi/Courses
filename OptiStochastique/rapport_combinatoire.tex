\documentclass{article}
%\usepackage[french]{babel}
\usepackage[utf8]{inputenc}
\usepackage{graphicx}
\usepackage{amsmath}
\usepackage{algorithm}
\usepackage[noend]{algpseudocode}

\makeatletter
\def\BState{\State\hskip-\ALG@thistlm}
\makeatother

\begin{document}
  
  \title{Combinatorary optimisation: \\
    \large Automatic paper assignement for the futur CS-DC'15 conference}
  \author{PALLAMIDESSI Joseph, ERSFELD Thomas}
  \maketitle
  
  \section{Introduction} % (fold)
  \label{sec:Intr}
    \paragraph{} % (fold)
    \label{par:}
    The goal of this project is to provide a fast and efficient automatic
    assignement of papers to reviewers for the CS-DC'15 conference. 
    This work is a continuation of the PPSN\cite{} publishe in the early 2000's. 
    Using a BRKGA\cite{}-like single objective genetic algorithm, the assignement will
    take care of avoiding assigning a paper issued from the same institution as a
    reviewer come from, assigning too many papers to a reviewer, and other similar
    use-case. All the parameters can be user defined to match specific needs. 

    %Gaussian
    % paragraph  (end)
  % section Intr (end)

  \section{Implementation} % (fold)
  \label{sec:Implementation}
    \subsubsection{Genome definition} % (fold)
    \label{ssub:Genome definition}
      
      \paragraph{} % (fold)
      \label{par:}
        The genome is defined as an array of papers, with each paper having an array of
        reviewers. This genome definition come from the need that a paper must have a
        certain number of reviewer (by default 3, but this value can be choosed by
        the user). The genome's array only contain identifiers of the problem's
        papers and reviewers, to limit the size of the genome, to provide more
        flexibility as well as to have as little as possible problem specific code
        in the implementation of the genetic opererators.
      % paragraph  (end)
      
      \paragraph{} % (fold)
      \label{par:}
        For the rest of this report, we will refer to the cells of the different
        arrays as \textit{genes}.
      % paragraph  (end)
    
    % subsubsection Genome definition (end)
    \subsubsection{Initialization} % (fold)
    \label{ssub:initialization}
      
      \paragraph{} % (fold)
      \label{par:}
      For each individual, we initialize each papers array (there are more papers
      arrays in the genome than papers in the problem) with random reviewer
      identifiers. On a large population, we assume that the random distribution is
      uniform and each possible reviewer is assigned some individuals. 
      % paragraph  (end)
      
       % subsubsection initialization (end)

    \subsubsection{Evaluation} % (fold)
    \label{ssub:Evaluation}
      
      \paragraph{} % (fold)
      \label{par:}

        The evaluation is the most specialized genetic operator of this project. 
        For each individual, before evaluating it, we \textit{decode} it in
        order to obtain the real data of their papers and reviewers.

        The fitness is then computed by following differents criteria:
        \\

        \begin{itemize}
          \item nbMatch: Number of match 
          \begin{itemize}
            \item Malus when no keywords is found
          \end{itemize}
          \item preference: The preference of the reviewer
          \begin{itemize}
            \item Malus when the reviewer is not willing to do the specific paper
            \item Bonus when the reviewer is willing to do the specific paper
          \end{itemize}

          \item instMalus: Institutions that have issued the paper
          \begin{itemize}
            \item Malus if a reviewer come from the same institution as the paper
          \end{itemize}
          \item sameMalus: The same reviewer is assigned twice or more to a paper 
        \end{itemize}

        
        fitness = $\sum_{t=1}^{nb_individuals} (\sum_{j=1}^{nb_reviewer_per_paper}( nbMatch +
        preference + instMalus + sameMalus)) $
        \\
        \\
      % paragraph  (end)

    % subsubsection Evaluation (end)
    
    \subsubsection{Crossover} % (fold)
    \label{ssub:Crossover}
      
      \paragraph{} % (fold)
      \label{par:}
      The crossover used is a simple one point one. A random paper position is
      choosed and the resulting individual is the mix between the first parent
      (until this random selected point) and the second parent after it.
      % paragraph  (end)
      
      \begin{algorithm}
      \caption{One point crossover}\label{pseudo1}
      \begin{algorithmic}[1]
      \Procedure{Crossover}{}
      \State $alpha\gets random(0.,nbPaper)$ 
      \For{i from 1 to nbPaper}
        
        \If{i < alpha}
          \State $Child.gene[i]\gets Parent1.gene[i]$
        \Else
          \State $Child.gene[i]\gets Parent2.gene[i]$
        \EndIf

      \EndFor
      \EndProcedure
      \end{algorithmic}
      \end{algorithm}

    
    % subsubsection Crossover (end)
    
    \subsubsection{Mutator} % (fold)
    \label{ssub:Mutator}
    
      \paragraph{} % (fold)
      \label{par:}
      Once again a simple method is used. For each reviewer identifier in an
      individual's ones, giving the mutation probability, we choose a random
      indentifier in the range of the possible reviewers.
      % paragraph  (end)

      \begin{algorithm}
      \caption{Random key swap mutator}\label{pseudo2}
      \begin{algorithmic}[2]
      \Procedure{Mutator}{}
      \State $alpha\gets random(0.,nbPaper)$ 
        \For{i from 1 to nbPaper}
          \For{j from 1 to nbReviewerPerPaper}
            \State $Child.gene[i].reviewer[j]\gets random(0., nbReviewer)$
          \EndFor
        \EndFor
      \EndProcedure
      \end{algorithmic}
      \end{algorithm}

      % subsubsection Mutator (end)
  % section Implementation (end)

  \section{Preliminary results} % (fold)
  \label{sec:Preliminaty Result}
    
    \paragraph{} % (fold)
    \label{par:}
      Given the ppsn data set, we observed an average fitness value that is
      negative.
    % paragraph  (end)
    
  % section Result (end)
   
   \section{Results} % (fold)
   \label{sec:Result}
   \paragraph{} % (fold)
   
     \label{par:}
       Here are the observed results, averaged on 30 runs:
     % paragraph  (end)
    
    \begin{figure}
    \begin{small} 
    \begin{tabular}{lrrr}
      Parameter &  \\
      \hline
      Nb of generations & 1000 \\
      Population size & 4096  \\
      Crossover probability & 1 \\
      Papers per reviewer & 3 \\
      Reviewers per paper & 3 \\
      Mutation probability & 0.01\\
      Surviving parents & 100\%  \\
      Surviving offspring & 100\% \\
      Elitism & Strong \\
      Elite & 1 \\ \hline
      \emph{Result} & -2.03e-03 \\
      \hline 
      \end{tabular}
      \caption{Parameters for one run}  
      \end{small}
      \end{figure}
   % section Result (end)

   \section{Further analysis and development} % (fold)
   \label{sec:section name}
   \paragraph{} % (fold)
   \label{par:}
   
   % paragraph  (end)
     We didn't test this implementation on GPGPU cards, and because of the
     simplicity of the genome, we can expect tremendious speedup. On a i5 core at
     3Ghz with only 2 core allocated, the optimisation took on average around 20 minutes. 
   % section section name (end)
  

\end{document}
